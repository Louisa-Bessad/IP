\documentclass[a4paper,12pt]{article}

%\usepackage[T1]{fontenc}
\usepackage{xltxtra}
\usepackage[francais]{babel}
\usepackage{fancyhdr}
%\usepackage[latin1]{inputenc}
\usepackage{epsfig}
\usepackage{calc}
\usepackage{url}
\usepackage{boxedminipage}

\usepackage{fontspec}
\setmainfont{Adobe Garamond Pro}

\usepackage[language=french]{csquotes}

\usepackage{titlesec}
\usepackage{graphicx}
\usepackage{algorithmicx}
\usepackage{algpseudocode}


%%%%%%%%%%%%%%%%%%%%%%%%%%%%%%%%%%%%%%%%%%%%%%%%%%%%%%%%%%%
%%%%%%%%%%%%%%%%%%%%%%%%%%%%%%%%%%%%%%%%%%%%%%%%%%%%%%%%%%%
%% Définitions à personnaliser 

%% Pour les noms, utilisez la première lettre du prénom suivi du 
%% nom de famille (première lettre majuscule, reste en minuscule).


%%%% Indiquer le nom de l'encadrant ci-dessous:

\def\nomEncad{Julien \textsc{Sopena}, Antoine \textsc{Blin}}

%% Si le projet est co-encadré indiquer les deux noms à la suite dans 
%% Le même champs


%%%% Indiquer les noms des étudiants participant ci-dessous:

\def\nomEtudA{Louisa \textsc{Bessad}}
\def\nomEtudB{Roberto \textsc{Medina}}

%% Si le projet est encadré par moins de 4 étudiants laissez
%% les variables inutiles vides


%%%% Indiquer la référence (numero) et le nom du sujet ci-dessous:

\def\refProjet{22} 
\def\titreProjetCourt{Temps-réel en multi-cœurs}
\def\titreProjetLong{Temps-réel en multi-cœurs~: problème de la contention mémoire}

%% Le titre court ne doit pas faire plus d'une vingtaine de caractère
%% résumez le à quelques mots essenciels


%%%% Indiquer le type de document et sa version ci-dessous:

\def\typeDoc{Rapport final}
 
%% - Rapport intermédaire
%% - Rapport final

%\let\origsec\section
%\renewcommand{\section}[1]{\newpage\origsec{#1}}



%%%%%%%%%%%%%%%%%%%%%%%%%%%%%%%%%%%%%%%%%%%%%%%%%%%%%%%%%%%
%%%%%%%%%%%%%%%%%%%%%%%%%%%%%%%%%%%%%%%%%%%%%%%%%%%%%%%%%%%
%% Définitions à ne pas modifier
 
%%%% ||| Mise en page verticale ||| 
\setlength{\voffset}{-1in} % a4:reste 297mm pour les 5 suivants:
\setlength{\topmargin}{15mm}         % avant l'en-tête
\setlength{\headheight}{20mm}        % hauteur de l'en-tête 
\setlength{\headsep}{10mm}            % entre l'en-tête et le corps
\setlength{\textheight}{220mm}       % hauteur du corps
\setlength{\footskip}{12mm}          % pied de page par rapport au corps 
%\setlength{\footlength}{2em}

%%%%% --- Mise en page horizontale ---
\setlength{\hoffset}{-1in} % a4:reste 210mm 
\setlength{\oddsidemargin}{25mm}     % entre hoffset et le corps
\setlength{\evensidemargin}{25mm}    % entre hoffset et le corps
\setlength{\marginparwidth}{0mm}     % largeur de la marge
\setlength{\marginparsep}{0mm}       % séparateur corps marge
\setlength{\textwidth}{160mm}        % largeur du corps

%\usepackage{fullpage}
%\setlength{\headheight}{20mm}        % hauteur de l'en-tête 
%\setlength{\headsep}{10mm}            % entre l'en-tête et le corps
%\setlength{\textheight}{200mm}
%\setlength{\footskip}{0mm}          % pied de page par rapport au corps 

\def\annee{2013-14}



%%%%%%%%%%%%%%%%%%%%%%%%%%%%%%%%%%%%%%%%%%%%%%%%%%%%%%%%%%%
%% Début du document

\begin{document}

\selectlanguage{francais}



%%%%%%%%%%%%%%%%%%%%%%%%%%%%%%%%%%%%%%%%%%%%%%%%%%%%%%%%%%%
%% Définition des en-têtes et pied de pages
\pagestyle{fancyplain}
%\fancyhead{}
%\fancyfoot{}
%
\fancyhead[L]{\textsc{Université Pierre et Marie Curie}\\
          Master Informatique\\ UE \textbf{PSAR} \annee \\ \nomEncad}
\fancyhead[C]{\textbf{Projet \refProjet\\\titreProjetCourt}}
\fancyhead[R]{\nomEtudA\\\nomEtudB\vspace{1em}}

\fancyfoot[L]{\includegraphics[width=3cm]{UPMC_sorbonne}}
\fancyfoot[C]{\textbf{\thepage/\pageref{fin}}}
\fancyfoot[R]{\typeDoc}

%\lhead[\fancyplain{}{\texttt{Université Pierre et Marie Curie}\\
%          Master Informatique\\ UE \textbf{PSAR} fév. \annee \\ \nomEncad}]
%      {\fancyplain{}{\textsc{Université Pierre et Marie Curie}\\
%          Master Informatique\\ UE \textbf{PSAR} \annee \\ \nomEncad}}
%\chead[\fancyplain{}{\textbf{Projet \refProjet\\\titreProjetCourt}}]
%      {\fancyplain{}{\textbf{Projet \refProjet\\\titreProjetCourt}}}
%\rhead[\fancyplain{}{\nomEtudA\\\nomEtudB}]
%      {\fancyplain{}{\nomEtudA\\\nomEtudB}}
%\lfoot[\fancyplain{}{\epsfig{figure=UPMC_sorbonne.eps,width=3cm}}]
%      {\fancyplain{}{\epsfig{figure=UPMC_sorbonne.eps,width=3cm}}}
%\cfoot[\fancyplain{}{\textbf{\thepage/\pageref{fin}}}]
%      {\fancyplain{}{\textbf{\thepage/\pageref{fin}}}}
%\rfoot[\fancyplain{}{\typeDoc}]
%      {\fancyplain{}{\typeDoc}}


%%%%%%%%%%%%%%%%%%%%%%%%%%%%%%%%%%%%%%%%%%%%%%%%%%%%%%%%%%%

      \begin{center}
        \begin{boxedminipage}{12cm}{
            \begin{center}
              ~\\\LARGE\textbf{\titreProjetLong}\\
              ~\\\large Encadrant: \textbf{\nomEncad,}\\
              ~\\\large Étudiants: \textbf{\nomEtudA, \nomEtudB}\\
              ~
            \end{center}
            }
        \end{boxedminipage}
      \end{center}

\vfill

%\newpage

\tableofcontents
\vfill
%\newpage

\newpage\section{Introduction}

\newpage\section{Problème de concurrence d'accès}

\newpage\section{Problème de la sous-réservation de BP mémoire}

\newpage\section{Conclusion face au sujet proposé}

\label{fin}
\end{document}
